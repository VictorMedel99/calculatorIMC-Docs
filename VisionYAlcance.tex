\documentclass[12pt,a4paper]{book}
\usepackage[utf8]{inputenc}
\usepackage[spanish]{babel}
\usepackage{amsmath}
\usepackage{amsfonts}
\usepackage{amssymb}
\usepackage{graphicx}
\usepackage{setspace}
\usepackage[left=2cm,right=2cm,top=2cm,bottom=2cm]{geometry}

\date{\today}

\begin{document}
 	\begin{titlepage}
	\begin{center}
	{\huge \textbf{Universidad Veracruzana}}\\
	\vspace{2cm}  
	{\Large {Visión y Alcance Del Proyecto}}\\
	\vspace{5mm}	
	{\Large {Calculadora IMC}}\\
	\begin{figure}[h]
		\centering
		\includegraphics[scale=0.10]{uvlogo}
	\end{figure}
	{\Large {Ingeniería de software}}\\
    \vspace{2cm}
	{\Large {Medel Ayohua Víctor Iván}}\\
	\vspace{3mm}	
	{\Large {Ramos García Luis Alberto}}\\
	\vspace{2cm}	
    \rule{8cm}{0.5mm} \\ \Large Vo.bo\\ 
	\end{center}
\end{titlepage}

\tableofcontents
\newpage

\section{Introducción}
\vspace{0.5 cm}
En el presente documento se identifican los aspectos correspondientes a la visión y alcance de este proyecto.\\
\\ El proyecto Calculadora-IMC es un proyecto donde se requiere poner a prueba conocimientos de administración de proyectos y pruebas de software con respecto al segundo parcial.
		
\section{Definiciones}
\vspace{0.5 cm}
\textbf {UML}: Unified Modeling Language, por sus siglas en inglés, la cual traduce
Lenguaje Unificado de Modelado.\\

\textbf {HTML}: HyperText Markup Language, por sus siglas en inglés, es un lenguaje
basado en etiquetas usado en el desarrollo web el cual brinda un estándar para
la definición de la estructura y para la definición de contenido de la página web
como: texto, imágenes y videos.\\

\textbf {Angular}: Framework para desarrollo de aplicaciones web desarrollado en TypeScript, de código abierto y mantenido por Google.\\

\textbf {Firebase}: Firebase de Google es una plataforma en la nube para el desarrollo de aplicaciones web y móvil. Está disponible para distintas plataformas (iOS, Android y web).\\

\textbf {TypeScript}: TypeScript es un lenguaje de programación libre y de código abierto desarrollado y mantenido por Microsoft. Es un superconjunto de JavaScript, que esencialmente añade tipos estáticos y objetos basados en clases.\\

\section{Historial de revisiones}
\vspace{0.5 cm}
\begin{table}[h!]
\centering
\begin{tabular}{|p{0.35\linewidth}|p{0.15\linewidth}|p{0.35\linewidth}|p{0.15\linewidth}|}
\hline
\textbf{Nombre}&\textbf{Fecha}&\textbf{Razón del cambio}&\textbf{Versión}
\\\hline
Víctor Iván Medel Ayohua&17/11/2020&Revisión inicial&v1.1\\\hline
Víctor Iván Medel Ayohua&17/11/2020&Definición de la estructura inicial del documento&v1.2\\\hline
Víctor Iván Medel Ayohua&18/11/2020&Redacción del capitulo 1&v1.3\\\hline
Víctor Iván Medel Ayohua&18/11/2020&Redacción de los requerimientos de la calculadora IMC&v1.4\\\hline
Víctor Iván Medel Ayohua&19/11/2020&Redacción del capitulo 2&v1.5\\\hline
Víctor Iván Medel Ayohua&19/11/2020&Redacción del capitulo 3,4 &v2.1\\\hline
\end{tabular}
\end{table}

\chapter{Requerimientos del negocio}
Los requerimientos de este proyecto proporcionan la base de la funcionalidad con la que deberá contar esta aplicación.\\
A partir de estos requerimientos nosotros identificamos los objetivos y tareas que los usuarios podrán realizar con esta herramienta.
\section{Escenario}
\vspace{0.5 cm}
Calculadora IMC esta destinada a brindar información a los usuarios con respecto a su indice de masa corporal y su peso, información que puede ser de ayuda para el usuario y así conocer su situación actual y tomar acciones en base a ello.
\section{Objetivos del negocio y criterios de éxito}
\vspace{0.5 cm}
Ayudar al usuario mejorando su experiencia con respecto a otras herramientas web que no desglosan la información del usuario y a su vez en muchos casos no puede ser comprendida correctamente.
\newpage
\vspace{0.5 cm}
\section{Necesidades del cliente o del mercado}
\vspace{0.5 cm}
Calculadora IMC mostrará la información al usuario según su estado actual en alguna grafica o tabla visualmente atractiva y entendible para cualquier tipo de usuario.\\

\textbf{Principalmente el sistema deberá cumplir los siguientes requisitos: }
\vspace{0.5 cm}
\begin{itemize}
\item \textit{Se podrá acceder a la aplicación desde cualquier navegador.}
\item \textit{El usuario podrá consultar información y calcular su IMC.}
\item \textit{El usuario podrá ver reflejada su información en alguna tabla o grafica.}
\item \textit{Deberá incorporar caracteres que sean visibles en la mayoría de los navegadores.}
\item \textit{Deberá Incorporar una interfaz de usuario sencilla estéticamente y completa según los datos requeridos para el calculo del IMC.}
\end{itemize}
\chapter{Visión de la solución}

\section{Declaración de la visión}
\vspace{0.5 cm}
Calculadora IMC es una aplicación que mostrará información de utilidad respecto al peso,estatura,edad y sexo de cada usuario, mostrando información como lo es el IMC, el peso según una grafica de rangos óptimos según su IMC antes calculado. 
\section{Características principales}
\vspace{0.5 cm}
\textbf {Requerimientos funcionales:}
\vspace{0.5 cm}
\begin{itemize}
\item \textit{Se podrá acceder al sistema desde cualquier navegador.}
\item \textit{La información se mostrara en gráficas.}
\item \textit{El usuario tendrá la capacidad de consultar mas información respecto a la salud a través de una liga a una pagina de salud como el IMMS.}
\end{itemize}
\textbf {Requerimientos no funcionales:}
\vspace{0.5 cm}
\begin{itemize}
\item \textit{El tiempo de carga de la información no deberá superar los 30 segundos.}
\item \textit{Presentara una interfaz sencilla y completa para el fácil ingreso de los datos.}
\item \textit{La interfaz de usuario tendrá colores suaves atractivos a la vista del usuario.}
\item \textit{El proyecto sera administrado en git.}
\item \textit{Utilizar kanban automatizados para la administración del repositorio.}
\item \textit{Cada pull request deberá ser autorizado por el líder del proyecto.}
\item \textit{Branch protegidos.}
\item \textit{Buen manejo de la documentación y control de versiones.}
\end{itemize}

\chapter{Contexto del negocio}
\section{Perfil de los involucrados}
Las partes involucradas son personas que participan activamente en un proyecto, que influyen en el resultado del proyecto. Los perfiles de las partes involucradas son:
\begin{table}[h!]
\begin{tabular}{|p{5 cm}|p{5 cm}|p{5 cm}|}
\hline
\textbf{Involucrado}&\textbf{Intereses principales}&\textbf{Limitaciones}
\\\hline

Víctor Iván Medel Ayohua &Realizar la Documentación y especificación de los documentos,contribuir con el desarrollo de los release y mockups de la UI para el desarrollo del proyecto.&Apegarse a los criterios de evaluación solicitados.\\\hline

Luis Alberto Ramos Garcia&Desarrollo de las funciones de cada release, diseñar los diagramas de clases,codificar pruebas para los 2 primeros release & Utilizar angular como framework y seguir los criterios de evaluación solicitados.\\\hline
\end{tabular}
\end{table}
\newpage
\section{Prioridades del proyecto}
\vspace{0.5 cm}
\begin{table}[h!]
\begin{tabular}{|p{5 cm}|p{5 cm}|p{5 cm}|}
\hline
\textbf{Prioridad}&\textbf{Objetivo}&\textbf{Rango de tiempo permitido}
\\\hline
Definir el Equipo de trabajo&constituir el equipo de un máximo de 2 integrantes& 13/11/2020.\\\hline
Planificación&Definir los objetivos del proyecto & 17/11/2020.\\\hline
Documentación&La documentación debe ser clara y estar bien organizada& 17/11/2020 - 20/11/2020.\\\hline
Desarrollo&El desarrollo cumpla con las funciones establecidas& 17/11/2020 - 22/11/2020.\\\hline
Pruebas&La aplicación deberá pasar las pruebas establecidos, se prueba la funcionalidad realizada& 17/11/2020 - 22/11/2020.\\\hline
Entregas&Entregar la aplicación el día 23/11/20200 & 20/11/2020 - 23/11/2020.\\\hline
\end{tabular}
\end{table}

\end{document}